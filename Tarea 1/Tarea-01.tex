\documentclass[10pt,letterpaper,fleqn]{article}

\usepackage[utf8]{inputenc}
\usepackage[spanish,es-nodecimaldot]{babel}
\usepackage{amsmath}
\usepackage{amssymb}
\usepackage{multicol}
\usepackage{graphicx}

\usepackage[dvipsnames]{xcolor}
\usepackage[most]{tcolorbox}

\usepackage{tabu}

\usepackage{pgfplots}
\pgfplotsset{width=10cm,compat=1.9}

\usepackage{mathtools}
\usepackage{tikz}
\usetikzlibrary{trees,positioning}

\usepackage[top=1in, bottom=1in, left=1in, right=1in]{geometry}


\begin{document}

\begin{titlepage}
    \centering

    {\scshape\LARGE Universidad Nacional Autónoma de México \par}

    \vspace{1cm}
    {\scshape\Large Facultad de Ciencias\par}
    \vspace{1.5cm}

    \begin{center}
        \includegraphics[scale=.1]{assets/img/logo.png}
    \end{center}

    \vspace{.8 cm}

    {\LARGE Tarea 1: \par}
    {\huge\bfseries Ejercicios \par}

    \vspace{0.5cm}
    \large{\itshape{Luis Erick Montes Garcia}} \small{ - 419004547}

    \vfill

    Trabajo presentado como parte del curso de
    \textbf{Teoría de Números I}
    impartido por el profesor \textbf{Javier Valdez Quijada}. \par
    \vspace{0.1cm}
    {\large Entrega 18 de Febrero del 2019 \par}
    \footnotesize{\textbf{Link al código fuente:} git@github.com:lemg98/Teoria-De-Numeros-I.git}
\end{titlepage}

    \begin{enumerate}

        %Ejercicio 1%
        \item \begin{enumerate}

            \item Determinar los valores de $n$ para los cuales $8n^3 + 12n^2 - 2n - 3$ es múltiplo de 5.\\
            \textbf{Solución: } Observamos los resultados de los números de $n\in [1,10]$ y en el siguiente renglón $n\in[2,20]$. \\
            (1,15) (2,105) (3,315) (4,693) (5,1287) (6,2145) (7,3315) (8,4845) (9,6783) (10,9177) \\ 
            (11,12075) (12,15525) (13,19575) (14,24273) (15,29667) (16,35805) (17,42735) (18,50505) (19,59163) (20,68757) \\ 
            Basta con que el resultado tenga 5 o 0 en sus unidades para que sea múltiplo de 5. Podemos notar que si las unidades de $n$ son $1,2,3,6,7,8$ entonces el resultado es múltiplo de 5. Daremos esta proposición.
            
            \item Demostrar su afirmación por inducción.\\
            \textbf{Solución: } Observamos
            \begin{equation*}
                \begin{split}
                    (2n - 1)(2n + 1)(2n + 3) & = (4n^2-1)(2n+3)\\
                                             & = 4n^2(2n+3) - (2n+3) \\
                                             & = 8n^3 + 12n^2 - 2n -3
                \end{split}
            \end{equation*}

            por tanto usaremos esta factorización para demostrar inductivamente nuestra proposición. \\
            Decimos que los factores son de la forma $(2n+x)$ con $x\in\{-1,1,3\}$, nos referiremos a estos en las siguientes afirmaciones, basta con que alguno de estos factores sea múltiplo de 5 para que el resultado lo sea. Procedemos a demostrar ambas afirmaciones.
            \begin{itemize}
                \item Podemos dar por ciertos los casos base por los resultados del inciso anterior.
                \item Suponemos que $n$ cumple que algún factor $(2n+x)$ es múltiplo de 5. Decimos que el siguiente número natural con las mismas unidades que $n$ es $n+10$. \\
                Observamos que $2(n+10)+x=2n+20+x=(2n+x)+5(4)$, ya que ambos son múltiplos de 5 entonces el resultado es múltiplo de 5. Por tanto si $n$ cumple la proposición $n+10$ también.
            \end{itemize}

            Una vez demostradas ambas afirmaciones podemos usar el principio de inducción matemática para concluir que si las unidades de $n$ son $1,2,3,6,7,8$ entonces $8n^3 + 12n^2 - 2n -3$ es múltiplo de 5.

        \end{enumerate}

        %Ejercicio 2%
        \item \begin{enumerate}
            
            \item Determinar los valores de n para los cuales $2^n > n^2 + 4n + 5$.\\
            \textbf{Solución: } Observamos los primeros 8 casos de n \\
            (1,2,10) (2,4,17) (3,8,26) (4,16,37) (5,32,50) (6,64,65) (7,128,82) (8,256,101)\\
            Determinamos que a partir de $n=7$ se cumple la desigualdad.

            \item Demostrar su afirmación por Inducción.\\
            \textbf{Solución: } Sea $n$ un número natural podemos decir que 
            \begin{equation*}
            \begin{split}
                n > 0 &\Rightarrow n^2 > 0 \\
                      &\Rightarrow n^2 + 4n > 2n \\
                      &\Rightarrow n^2 + 4n +(4+1) > 2n + (4+1)  
            \end{split}
            \end{equation*} llamaremos (*) a esta afirmación.
            Demostraremos las siguientes dos afirmaciones.\\
            \begin{itemize}

                \item Para $n=7$ se cumple la desigualdad como vimos en el inciso anterior.
                \item Suponemos que se cumple para $n$. Partimos de 
                \begin{equation*}
                \begin{split}  
                          2^n &> n^2 + 4n + 5 \text{ por hipótesis},\\
                    2^n + 2^n &> (n^2 + 4n + 5) + (n^2 + 4n +5), \\
                    2^n + 2^n &> (n^2 + 4n + 5) + 2n + (4+1) \text{ por (*)}, \\
                        2(2^n)&> (n^2 + 2n + 1) + (4n + 4) + 5,\\
                      2^{n+1} &> (n+1)^2 + 4(n+1) + 5 
                \end{split}
                \end{equation*}
                Por tanto si $n$ cumple la desigualdad $n+1$ también la cumple.         
            \end{itemize}

            Demostradas ambas afirmaciones podemos usar el Principio de Inducción Matemática para concluir que si $n > 6$ entonces $2^n > n^2 + 4n+ 5$.

        \end{enumerate}

    \end{enumerate}            

\end{document}
